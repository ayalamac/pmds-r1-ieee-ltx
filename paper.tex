\documentclass[journal]{IEEEtran}

\usepackage{tikz}
\usepackage{blindtext}
\usepackage[hidelinks]{hyperref}

\ifCLASSINFOpdf
\else
\fi


\hyphenation{op-tical net-works semi-conduc-tor}
\renewcommand\refname{Referencias}

\begin{document}

%\title{Diseñando la Agilidad en \textit{Machine Learning}:\\ Un Nuevo Marco de Trabajo para el Desarrollo de Software}

\title{CRISP-VML: Propuesta de desarrollo ágil para \textit{Visual Machine Learning} en entornos comerciales}

\author{Sara~Herrera, Andrés~Ayala, Julio~Franco y Martín~Villegas, \textit{Universidad EAFIT}}

\markboth{EAFITPMDS, Reto #1, Febrero~2024}%
{Shell \MakeLowercase{\textit{et al.}}: Bare Demo of IEEEtran.cls for IEEE Journals}

\maketitle

\begin{abstract}
This paper delves into the adaptation of agile and structured methodologies for software development in the Machine Learning (ML) domain, focusing on the successful integration of CRISP-DM. Through detailed analysis, we identify how iteration, collaboration, and a focus on robust design and architecture principles can overcome the unique challenges presented by ML. We propose a series of methodological improvements and development practices that demonstrate improved efficiency and greater adaptability to changes in requirements and data. Our findings reveal critical strategies for implementing scalable, efficient, and reproducible ML solutions, offering valuable insights for researchers and practitioners alike.
\end{abstract}

\begin{IEEEkeywords}
Software Development, Machine Learning, CRISP-DM
\end{IEEEkeywords}


\IEEEpeerreviewmaketitle


\section{Dominio}
\IEEEPARstart{M}{}achine Learning (ML) es una rama de la Inteligencia Artificial (IA) caracterizada por su habilidad de permitir que las máquinas aprendan patrones a partir de los datos sin una programación explícita \cite{Russell2021-eg}. Este dominio está en un estado de madurez tal que ha permitido a las organizaciones mejorar significativamente la eficiencia de los empleados, fomentar la innovación y reducir costos mediante su adopción creciente \cite{ibmWhatMachine}.\\

Desde sus inicios en la década de 1950s con el primer juego de IBM y el reconocimiento de las primeras capacidades de auto-aprendizaje en procesos de cómputo, los modelos de ML han evolucionado desde experimentos sencillos hasta herramientas poderosas y disruptivas, con capacidades de aprendizaje y adaptación que tratan de imitar la inteligencia humana con resultados cada vez más sorprendentes.\\

Hoy en día, el uso de modelos de ML en los procesos de innovación y transformación digital está revolucionando a pasos acelerados la manera en que operan las empresas de casi todas las industrias existentes.\\

En términos de principios de diseño y arquitectura, los proyectos de ML siguen lineamientos que garantizan la robustez, la escalabilidad y la eficiencia. Estos principios incluyen el diseño modular de los sistemas para permitir la experimentación y la iteración rápida, así como la implementación de pipelines de datos eficientes para el manejo y procesamiento de grandes volúmenes de datos \cite{10081336}. Además, se enfatiza la reproducibilidad de los resultados, lo que implica la necesidad de mantener un registro detallado de los experimentos y sus parámetros. Se adopta una arquitectura que puede escalar de manera elástica con la carga de trabajo y que es capaz de integrarse con tecnologías de almacenamiento y procesamiento de datos existentes.\\

El ciclo de vida de las soluciones de ML en este dominio sigue una naturaleza iterativa y evolutiva, apoyándose en dinámicas de retroalimentación continua. Este proceso no es lineal, sino que a menudo requiere volver a etapas anteriores a medida que se obtiene más información y se refina el entendimiento del problema y los datos. La evaluación continua durante todas las fases asegura que el modelo se ajusta a las necesidades y que cualquier cambio en los datos o en los objetivos del negocio puede incorporarse de manera eficiente \cite{10081336}.\\

Los avances de la computación en la nube y el continuo lanzamiento de nuevas tecnologías en el campo de la inteligencia artificial facilitan que grandes corporaciones, \textit{start-ups} e individuos tengan mayor facilidad para crear proyectos de ML. Sin embargo, la sobre-saturación de información, la necesidad de recopilación, identificación y preparación de grandes volúmenes de datos en poco tiempo y, los esfuerzos para trabajar de manera colaborativa y sinérgica, representan los mayores retos que desafían el éxito de este tipo de proyectos.\\

\section{Selección de enfoque CRISP-DM}

CRISP-DM \textit{(Cross-Industry Standard Process for Data Mining)} es un modelo de proceso robusto y bien establecido que proporciona una estructura detallada para llevar a cabo proyectos de minería de datos y Machine Learning. Desarrollado por un consorcio de empresas con el objetivo de estandarizar la forma en que se abordan los proyectos de datos, CRISP-DM ha demostrado ser flexible y aplicable en diversos contextos de industria y negocio \cite{datasciencepmWhatCRISP}.\\

La elección de la metodología CRISP-DM como base para trabajar proyectos de ML se fundamenta en ser el marco de trabajo más apropiado para desarrollar un proyecto de \textit{data mining}, etapa crucial de proyectos que involucran grandes volúmentes de datos. Si bien CRISP-DM es bastante conocido para abordar este tipo de problemas de ML, también se integra con enfoques ágiles para permitir una mayor flexibilidad y respuesta rápida a los cambios \cite{datasciencepmWhatCRISP}. Esto es particularmente útil en ML, donde los modelos pueden necesitar ser reentrenados y ajustados continuamente a medida que se dispone de nuevos datos.\\

Su relación con la ingeniería de software es intrínseca, dado que CRISP-DM ofrece un marco de trabajo que se alinea bien con las etapas de desarrollo de software, especialmente en proyectos donde el procesamiento de datos y la generación de modelos de ML son centrales \cite{Forward2008}. Se enfoca en la comprensión y preparación de datos, el modelado y la evaluación, que son fundamentales en el desarrollo de software para aplicaciones de datos y ML. En el contexto de la ingeniería de software, CRISP-DM ayuda a guiar el proceso de desarrollo desde la comprensión del problema hasta la implementación de la solución, asegurando que los proyectos se completen de manera sistemática y reproducible.\\

Al abordar proyectos en el contexto de ML, los equipos adoptan un enfoque iterativo y adaptativo. Se prefiere el manejo de proyectos que puedan acomodar el rápido cambio y la evolución de los datos y los requisitos del negocio, lo que incluye la adopción de metodologías como CRISP-DM \cite{datasciencepmWhatCRISP}. Esta metodología proporciona un marco estructurado para guiar el desarrollo de proyectos de ML a través de sus fases de comprensión del negocio, análisis de datos, preparación, modelado, evaluación y despliegue.\\

La Figura \ref{fig:cripsDM}, ilustra el flujo de trabajo en el enfoque CRISP-DM, el cual se describe a continuación:\\

\begin{itemize}
    \item \textbf{Comprensión del Negocio}: Definir los objetivos del proyecto y los requisitos desde una perspectiva de negocio.
    \item \textbf{Comprensión de los Datos}: Iniciar una recopilación de datos y proceder con actividades para familiarizarse con los datos.
    \item \textbf{Preparación de los Datos}: Construir el conjunto de datos final a partir de los datos iniciales para alimentar los modelos de ML.
    \item \textbf{Modelado}: Seleccionar y aplicar técnicas de modelado para descubrir patrones o predecir fenómenos.
    \item \textbf{Evaluación}: Evaluar el modelo para asegurarse de que satisface los objetivos del negocio.
    \item \textbf{Despliegue}: Implementar el modelo en un entorno de producción y monitorear su desempeño.\\
\end{itemize}

\begin{figure}[h!]
    \centering
    \includegraphics[width=0.45\textwidth]{img/crisp_process.png}
    \caption{Ciclo de vida de minería de datos en CRISP-DM \cite{ibmConceptosBxE1sicos}.}
    \label{fig:cripsDM}
\end{figure}

Las técnicas y herramientas propuestas en CRISP-DM son variadas y dependen de la fase del proceso. Por ejemplo, para la comprensión de los datos y la preparación, se pueden utilizar herramientas de análisis exploratorio como Python con bibliotecas como Pandas y Matplotlib, o entornos de software como R. Para el modelado, se utilizan herramientas y plataformas que soportan algoritmos de ML como scikit-learn, TensorFlow, y Keras. Para el despliegue, se pueden emplear soluciones de contenedores como Docker y plataformas de orquestación como Kubernetes, así como servicios en la nube como AWS, Google Cloud o Azure que ofrecen capacidades de ML como servicio.\\

En cuanto a los lenguajes de modelado, CRISP-DM es agnóstico en términos de lenguajes de programación, pero en la práctica, Python y R son los más utilizados debido a su amplia gama de librerías y frameworks orientados a la ciencia de datos y ML. Además, los lenguajes de modelado específicos como UML (Unified Modeling Language) pueden ser útiles en la fase de comprensión del negocio para modelar los procesos y los requisitos.\\

Los métodos utilizados en CRISP-DM abarcan desde técnicas estadísticas y matemáticas para la evaluación y el modelado, hasta métodos de gestión de proyectos para la planificación y el despliegue. Se hace hincapié en métodos iterativos y flexibles que permiten la adaptación y la mejora continua del modelo y del proceso de desarrollo del software.\\

\section{Diseño modelo de trabajo}

La siguiente propuesta esta fundamentada en el trabajo expuesto en \cite{Cret2013} sobre la creación de una taxonomía de software.

\begin{figure}[h!]
    \centering
    \includegraphics[width=0.5\textwidth]{img/espina.png}
    \caption{Caption}
    \label{fig:espina}
\end{figure}

\subsection{Ciclo}

Para un dominio como Machine Learning, un modelo iterativo de manera local, regional y global con ciclos cortos (1-3 meses) es esencial debido a la naturaleza experimental y evolutiva de la construcción de modelos. Repetir un conjunto de tareas de diseño, desarrollo y evaluación permite ajustar y mejorar continuamente los modelos tal como lo expone el enfoque.\\

Se propone además un ciclo que permita iterar en etapas anteriores según la necesidad sin una definición concreta para su realización, dando flexibilidad en dichos escenarios.\\

La identificación y el orden de las etapas son cruciales en el desarrollo de software para Machine Learning, donde cada actividad --como la preparación de datos o la selección de algoritmos-- es fundamental. La paralelización de etapas no es aconsejable debido a la naturaleza de pasos sistemáticos del dominio junto con un modelo orientado a las actividades, se asegura que todos los aspectos críticos del desarrollo de ML sean abordados sistemáticamente, en línea con la fase de comprensión del negocio y los datos.\\

\subsection{Colaboración} \label{colaboracion}

La colaboración interna en equipos de características multifuncionales y/o jerárquicas es vital para afrontar la complejidad del Machine Learning, donde diferentes habilidades y conocimientos deben converger para crear soluciones efectivas. Se proponen los roles Data Analyst (DA), Business Analyst (BA), Data Scientist (DS), Machine Learning Engineer (MLE), Product Owner (PO). Haciendo especial referencia que aunque la figura de la estructura base \cite{Cret2013} separe en colaboración interna/externa, el equipo propuesto no separa los roles ya que todos convergen a un mismo objetivo. Esto se alinea con la estructura propuesta por CRISP-DM, que requiere colaboración y coordinación entre expertos en negocios, datos y modelado.\\

El diseño centrado en el usuario y en el uso garantiza que las soluciones de Machine Learning cumplan con las necesidades y expectativas del usuario final y su funcionalidad en el ámbito requerido.

\subsection{Artefactos}

En el contexto de Machine Learning, los equipos se encuentran con diferentes artefactos internos para el correcto seguimiento del progreso y entrega continua. Los artefactos no ejecutables como los análisis de riesgo y los modelos son cruciales durante las iteraciones. Estos documentos sirven como registros vitales para entender la evolución del proyecto.\\

La formalización de artefactos asegura la claridad y la consistencia en la comunicación del proyecto. Sin embargo, dado que el ML es tan amplio en sus aplicaciones se sugiere una formalización semi-formal para sus ejecutables e informal en sus no ejecutables y así promover una mejor comprensión de los artefactos y su trazabilidad.\\

Para sus artefactos entregables ejecutables al usuario se tiene un modelo final de ML, el cual cubrirá la necesidad inicial, documentación y entregas semi-formales al usuario en donde se dará por finalizado oficialmente el proyecto.

\subsection{Uso recomendado}

El esquema propuesto para la metodología se recomienda para proyectos de tamaño medio-largo conformado por equipos pequeños (6-10 personas) dada la complejidad e incertidumbre que conlleva un proyecto de Machine Learning y el entrenamiento de modelos como salida final. Siguiendo el enfoque escogido y buscando facilidades para los equipos, se recomienda definir algunos procedimientos para evaluación de riesgos y cambios de requerimientos a lo largo del desarrollo, este modelo se define para aplicaciones con modelos de ML con alto insumo de data y consumo de imágenes.\\

La experiencia del equipo es importante en ML debido a la complejidad técnica y la necesidad de innovación. Métodos que reconocen la importancia de contar con expertos o líderes experimentados se alinean bien con la necesidad de liderazgo y conocimiento especializado en todas las fases. En caso de no contar con la posibilidad de todo un equipo con una alta experticia, se recomienda el uso de trabajo en pares para los roles del área técnica propuestos en la sección \ref{colaboracion}.\\

\subsection{Madurez}

La capacidad de un modelo para validar el proceso de diseño es esencial. Las métricas que evalúan la aplicación de un modelo de proceso no solo mejoran la calidad del desarrollo, sino que también proporcionan retroalimentación para la mejora continua, lo cual refleja la fase de evaluación del enfoque CRISP-DM. El proceso del modelo contiene grandes partes de formalización semi-formal permitiendo el uso de un lenguaje natural bien estructurado, se busca además que el enfoque este muy en sintonía con el método propuesto por lo cual tendría una muy buena difusión en el medio.\\

\subsection{Flexibilidad}

La adaptabilidad de la metodología a las necesidades específicas del proyecto y a los factores humanos es una característica valiosa, especialmente en ML. Es por ello, que el modelo propuesto permite variaciones definidas en algunas etapas del proceso, permitiendo extensiones o reducciones brindando herramientas y guías en cada caso. La finalización del modelo se da con la entrega completa del producto final de ML al usuario.\\

Esta forma de trabajo se puede detallar en un documento completo con varios aspectos o detalles específicos sobre algunos algoritmos normalmente usados en ML y aspectos complementarios del método CRISP-DM.\\

\section{Conclusiones}
En conclusión, este trabajo subraya la importancia crítica de elegir y adaptar metodologías de desarrollo de software que se ajusten a la naturaleza dinámica y compleja del Machine Learning. Las contribuciones clave incluyen la validación de CRISP-DM como un marco efectivo cuando se complementa con prácticas ágiles, y la identificación de estrategias para mejorar la colaboración multidisciplinaria y la adaptabilidad del proceso de desarrollo. Las limitaciones de nuestro estudio sugieren áreas para futuras investigaciones, particularmente en la exploración de métodos para integrar más profundamente la retroalimentación del usuario final en el ciclo de vida del desarrollo de ML. Recomendamos que futuros trabajos consideren el impacto de las tecnologías emergentes en la adaptación de metodologías de desarrollo de software para ML, con el objetivo de fomentar la innovación y mejorar la toma de decisiones basada en datos.


\bibliographystyle{IEEEtran}
\bibliography{bibtex/bib/IEEEexample}

\end{document}


